\documentclass[a4paper,12pt]{report}
\usepackage[utf8]{inputenc}
\usepackage[activeacute,spanish]{babel}
% Remove hyphenation
\tolerance=1
\emergencystretch=\maxdimen
\hyphenpenalty=10000
\hbadness=10000
\begin{document}
\title{Derecho Civil I}
\author{Alberto Corsín Lafuente}
\date{\today}
\maketitle
\tableofcontents
\chapter{Tema I}
\section{El concepto de derecho civil}
\subsection{Introducción}

No se puede dar una definición precisa del derecho que satisfaga las diversas
líneas del pensamiento y resulte admisible para la generalidad de autores. Ya
desde Immanuel Kant (s XIX) los juristas buscan una definición del concepto de
Derecho.

Para los tratadistas franceses del siglo XIX el Derecho Civil venía representado
por el Código Civil de 1804 (escuela exegesis). Los autores alemanes de
comienzos del siglo XX lo identificaron con el contenido del Burgerliches
Gesetzbuch, código alemán de 1896 (BGB) vigente desde principios de 1990. Sin
embargo, ninguno de los dos grupos llegó a formular una definición de aquello
que había de ser objeto de sus normas.

Semejante identificación material entre los Códigos y el Derecho Civil resulta
inaceptable de forma mayoritaria en los inicios del siglo XXI por razones
evidentes. Por importantes que puedan ser los Códigos Civiles, es obvio que el
Derecho Civil no quedó petrificado en ellos, sino que la legislación posterior y
las coordenadas sociales en general han diversificado su contenido, cuando no
han roto algunos de los principios básicos en que se asentaba la regulación
codificada. El Derecho Civil es algo más que un código.

\subsection{El planteamiento historicista y concepción apriorística}

Las teorías contemporáneas se dividen en historicistas y apriorísticas.

Las \textbf{teorías historicistas} recalcan el carácter histórico y evolutivo
del Derecho Civil. Recalcan la mutabilidad y evolución de las instituciones
jurídicas sin negar su relación con el Derecho Natural. Según estas teorías el
Derecho Civil es distinto en cada época.

Las \textbf{teorías apriorísiticas} o racionalistas puros hacen especial
hincapié en la nota de permanencia del derecho civil desde una perspectiva meta
histórica. Según esta línea de pensamiento los principios generales o
instituciones del Derecho Civil son siempre iguales, constantes. Cambia la forma
de regular, no los principios y se recalca la conexión entre Derecho Civil y
Derecho Natural. Por ejemplo: el principio de libertad de la persona, principio
de autonomía de la voluntad, etc. Estas teorías son minoritarias.

\subsection{La superación de la contraposición entre historicismo y apriorismo}

Doctrinalmente hablando resulta sumamente difícil encontrar manifestaciones
puras y radicales del apriorismo o del historicismo y abundan las posturas
intermedias. Lo que normalmente pretenden subrayas los aprioristas no es la
existencia permanente de un conjunto normativo llamado Derecho Civil, sino la
conexión de las históricas formas del mismo con principios que trascienden a un
ordenamiento jurídico determinado. O dicho con mayor precisión, los principios
que normalmente se reconocen como tributarios del Derecho Natural.

A su vez, los historicistas subrayan el aspecto más inmediato de la mutabilidad
y evolución de las instituciones jurídicas, sin que ello implique el
desconocimiento o negación de las relaciones de las mismas con los presupuestos
o principios del denominado Derecho Natural.

\begin{quote}
    \textit{En el medio está la virtud.}. Moral a Nicomano - Aristóteles.
\end{quote}

\begin{quote}
    \textit{Aunque el derecho civil (las instituciones que lo forman) se base
    directamente en el Derecho Natural, el ajuste entre aquel término y su
actual contenido ha sido determinado por circunstancias históricas}. Prof. F. de
Castro.
\end{quote}
\section{La materia propia del derecho civil}
\subsection{Derecho civil como derecho de la persona}

La form codificada del Derecho Civil ha sido el punto crítico de su evolución,
un instrumento. El núcleo central del Derecho Civil viene representado por la
\emph{persona} en sí misma considerada, en su dimensión \emph{familiar} y en sus relaciones
\emph{patrimoniales}, como revela la mera contemplación del índice sistemático
de cualquiera de los Códigos Civiles.

La propia estructura del Código Civil español demuestra lo anterior. Analizando
el contenido del mismo, las materias sobre las que recae su regulación serían
las siguientes:

\begin{itemize}
    \item{Vigencia y efecto de las normas jurídicas}
    \item{Delimitación del ámbito de poder jurídico de las personas y su
        relación con un grupo especial de otras personas que les son
    especialmente próximas por razón del nexo biológico o adoptivo entre ellas
existente (familia)}
    \item{Categorías de bienes que pueden ser objeto de tráfico; poder que las
        personas pueden ostentar sobre dichos bienes; modos de circulación de
    dichos bienes y reglas de transmisión de tales bienes (herencia)}
\end{itemize}

La materia contemplada en el primero de los apartados se refiere a cuestiones
generales de \textit{fuentes del Derecho} y de aplicación y eficacia de las
normas jurídicas que no pueden ser consideradas como exclusivas del Derecho
Civil aunque se integraron en los Códigos Civiles por razones históricas.

Las situaciones típicas que pueden configurarse como contenido de las diversas
formas históricas del Derecho Civil han sido tradicionalmente individualizadas
en la persona, en la familia y en el patrimonio:

\begin{itemize}
    \item{La \textbf{persona} en sí misma considerada, en cuanto sujeto de
        derecho, sin tener en cuenta cualesquiera otros atributos,
    características o situaciones sociales. Cuando se considera a la persona
como el empresario, entrará en juego el derecho mercantil, cuando se considera
como votante, el derecho constitucional, el electoral, etc.}
    \item{La \textbf{familia} como grupo humano básico, necesitado de una
        regulación que encuadre los derechos y deberes recíprocos entre sus
    miembros y de éstos con el resto de la comunidad}
    \item{El \textbf{patrimonio}, conjunto de bienes, derechos y obligaciones de
        cualquier persona con capacidad para adquirir y transmitir bienes.
    También, los instrumentos básicos de intercambio económico (los contratos) y
los mecanismos de transmisión a los familiares a través de la herencia, etc.}
\end{itemize}

Esta partición de contenidos se corresponde con realidades y situaciones de
permanente existencia. El Derecho Civil tiene un marcado carácter social, lo que
conlleva una tensión constante entre el grupo social políticamente organizado,
considerado en su conjunto, y los individuos que lo integran, considerados como
personas en sí mismas.

El problema principal es el marco de libertad y de autonomía del indivíduo
frente al grupo social políticamente organizado. Individuo vs Grupo. También
existe otro problema si la familia se adscribe al individuo o al grupo social.

Se puede conceptuar el Derecho Civil como el \textit{derecho de la personalidad
privada, que se desenvuelve a través de la familia, sirviéndose para sus propios
fines de un patrimonio y asegurando su continuidad a través de la herencia
(Profesor A. Cossio)}.

\section{La codificación y los derechos forales}
\subsection{La codificación en general}

Hacia medidados del siglo XVIII se produce un intento generalizado en toda
Europa de realizar una sistematización del Derecho sobre patrones diversos.
Hasta entonces se había calificado como \emph{codex} o \emph{códice} a un
conjunto de folios en forma de libro, cosido por el lomo y que tenía por objeto
recopilar conjuntos muy heterogéneos de cuestiones o máximas jurídicas sin
criterio determinado. A veces se ordenaban cronológicamente, a veces según la
fuente y otras de forma completamente anárquica.

A partir de este momento la palabra \emph{código} pasa a tener un significado
preciso y representa un ideal a alcanzar para todas las naciones europeas y
muchas sudamericanas.

Dicho periodo puede considerarse abierto con la publicación del Código Civil
francés en 1804y cerrado con la aprobación del Código Civil Alemán de 1896
(Bürgerliches Gesetzbuch o BGB). En ese periodo se racionalizó la materia
jurídica clarificando el sistema jurídico. Se estructura un sistema normativo
único que se aplicará a la generalidad de los ciudadanos (Francia) o súbcitos.

Esta codificación está caracterizada por Claridad, Sistematización, Igualdad
política (teniendo en cuenta las estructuras sociopolíticas del momento) y
Uniformidad Jurídica.

\subsection{La codificación civil española}
\subsubsection{El fracasado proyecto de 1851 y la publicación de las leyes especiales}

El bloque de legislaciones históricas superpuestas era de tal naturaleza que el
fenómeno codificador se consideraba como un necesario punto de partida para la
construcción de la España del siglo XIX. Todas las constituciones decimonónicas
incliyen dentro de su articulado la aspiración a la codificación del Derecho
patrio.

Dicha aspiración resultó relativamente pacífica en relación con algunas materias
como el Derecho Mercantil, que ya tomara cuerpo codificado en 1829. La
codificación civil, sin embargo, resultó mucho más problemática por la tensión
existente entre el Derecho Común y los Derechos Forales, entre otras razones.

El verdadero punto de partida de la codificación civil española viene
representado por el \textbf{Proyecto de Código Civil de 1851}, comunmente
llamado \emph{Proyecto isabelino}. Dicho trabajo nace como uno de los primeros
frutos de la recién creada Comisión General de Códigos, según Real Decreto del
19 de agosto de 1842, que actualmente se llama Comisión General de Codificación
y que dependía del Ministerio de Justicia.

Dicho proyecto se caracterizaba por dos características que provocaron su
fracaso:

\begin{itemize}
\item{Ser notoriamente afrancesado, siguiendo muy de cerca los patrones propios del Código Civil francés}
\item{Unificaba la legislación civil española, eliminando los Derechos Forales}
\end{itemize}

La necesidad de encarar la actualización de la legislación civil era evidente y
en las decadas siguientes se fue desgranando paulatinamente la aprobación de
leyes importantísimas que deberían haber sido incorporadas al Código Civil como
la Ley Hipotecaria, la Ley de Matrimonio Civil, la Ley de Registro Civil o la
Ley de Propiedad Intelectual. Tales leyes reciben la adjetivación de
\textbf{especiales} en cuanto se considera que los aspectos comunes o
fundamentales de ellas deberían ser recogidos en el Código Civil.

La calificación de \emph{especiales} resalta \emph{ab initio (desde el principio)} la idea de que son
leyes extracodificadas, leyes especiales en contraposición a las leyes generales
que están en el Código Civil.

\subsubsection{El código civil}

En enero de 1880, el Ministro de Justicia Álvarez Bugallal, insta a la comisión
de Códigos a que en el plazo de un año redacte el Código sobre la base del
Proyecto de 1851, incorporando al efecto juristas de los territorios forales.

Al mes siguiente, Manuel Alonso Martínez trata de impulsar la tarea codificadora
recurriendo a la idea de la \textbf{Ley de Bases (los principios, fundamentos o
presupuestos fundamentales)}. Se presentaría a las Cámaras
legislativas una Ley en la que se contuvieran los principios y fundamentos a
desarrollar en el Código Civil, mientras que en la redacción del texto
articulado quedaría encomendada a los organismos técnicos. Rechazado en el
Congreso, se vio obligado a presentarlo ante la Cámara por libros, esto es, por
partes.

Con ligeros retoques, volvió a insistir en la idea de la Ley de Bases en 1885,
la cual tras la consiguiente tramitación parlamentaria fue aprobada como Ley en
1888.

Siguiendo las bases establecidas por dicha Ley, la Comisión de Códigos llevo a
cabo su misión de redactar el texto articulado de forma algo atropellada y
con recortes de última hora.  El texto del Código Civil se publicó finalmente en la \emph{Gaceta}
(hoy B.O.E.) de 25, 16 y 27 de julio de 1889.

\subsubsection{La evolución posterior de la legislación civil}

Desde su publicación hasta la fecha, el texto articulado del Código Civil ha
sido objeto de numerosas reformas, aunqeu la mayor parte del mismo ha resistido
bien el paso del tiempo y sigue fiel a los textos originarios.

Entre las reformas habidas, las fundamentales y más profundas son las debidas a
la necesidad de adecuar el contenido del Código a la Constitución española de
1978, sobre todo en lo referente a la igualdad entre hombres y mujeres y a la
igualdad entre hijos matrimoniales y extramatrimoniales.

Otro bloque de disposiciones modificativas del texto articulado del Código Civil
responde a razones meramente técnicas. Sin duda la más importante es la Ley de
1973, en cuya virtud se dio nueva redacción al \emph{Título preliminar} del
Código Civil.

\subsection{Los derechos forales y la llamada cuestión foral}

Durante los siglos XVIII y XIX existía en España una cierta diversidad de
regulaciones civiles pues Aragón, Navarra, Mallorca, Cataluña y las Provincias
Vascongadas mantenían reglas propias en material civil, sobre todo en lo
referido a la familia y a la herencia. Sin embargo, la codificación requería la
unificación legislativa en toda España, a lo que los juristas forales se
negaban. Este fue uno de los motivos por los que no salió adelante el
\emph{proyecto isabelino}, al pretender abrogar (derogar) los Derechos Forales.

Cuando se aprueba el Código Civil a finales del siglo XIX no hay consenso y nade
la "cuestión foral". Una vez aprobado el Código Civil, este se aplica a la mayor
parte del territorio nacional, mientras que en los Territorios Forales (Aragón,
Navarra, Mallorca, Cataluña, Provincias Vascongadas y se añade Galicia en 1880)
rigen disposiciones de naturaleza civil propias, de diferente alcance, extensión
y significado.
