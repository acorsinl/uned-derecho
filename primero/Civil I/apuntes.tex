\documentclass[a4paper,12pt]{report}
\usepackage[utf8]{inputenc}
\usepackage[activeacute,spanish]{babel}
% Remove hyphenation
\tolerance=1
\emergencystretch=\maxdimen
\hyphenpenalty=10000
\hbadness=10000
\begin{document}
\title{Derecho Civil I}
\author{Alberto Corsín Lafuente}
\date{\today}
\maketitle
\tableofcontents
\chapter{Tema I}
\section{El concepto de derecho civil}
\subsection{Introducción}

No se puede dar una definición precisa del derecho que satisfaga las diversas
líneas del pensamiento y resulte admisible para la generalidad de autores. Ya
desde Immanuel Kant (s XIX) los juristas buscan una definición del concepto de
Derecho.

Para los tratadistas franceses del siglo XIX el Derecho Civil venía representado
por el Código Civil de 1804 (escuela exegesis). Los autores alemanes de
comienzos del siglo XX lo identificaron con el contenido del Burgerliches
Gesetzbuch, código alemán de 1896 (BGB) vigente desde principios de 1990. Sin
embargo, ninguno de los dos grupos llegó a formular una definición de aquello
que había de ser objeto de sus normas.

Semejante identificación material entre los Códigos y el Derecho Civil resulta
inaceptable de forma mayoritaria en los inicios del siglo XXI por razones
evidentes. Por importantes que puedan ser los Códigos Civiles, es obvio que el
Derecho Civil no quedó petrificado en ellos, sino que la legislación posterior y
las coordenadas sociales en general han diversificado su contenido, cuando no
han roto algunos de los principios básicos en que se asentaba la regulación
codificada. El Derecho Civil es algo más que un código.

\subsection{El planteamiento historicista y concepción apriorística}

Las teorías contemporáneas se dividen en historicistas y apriorísticas.

Las \textbf{teorías historicistas} recalcan el carácter histórico y evolutivo
del Derecho Civil. Recalcan la mutabilidad y evolución de las instituciones
jurídicas sin negar su relación con el Derecho Natural. Según estas teorías el
Derecho Civil es distinto en cada época.

Las \textbf{teorías apriorísiticas} o racionalistas puros hacen especial
hincapié en la nota de permanencia del derecho civil desde una perspectiva meta
histórica. Según esta línea de pensamiento los principios generales o
instituciones del Derecho Civil son siempre iguales, constantes. Cambia la forma
de regular, no los principios y se recalca la conexión entre Derecho Civil y
Derecho Natural. Por ejemplo: el principio de libertad de la persona, principio
de autonomía de la voluntad, etc. Estas teorías son minoritarias.

\subsection{La superación de la contraposición entre historicismo y apriorismo}

Doctrinalmente hablando resulta sumamente difícil encontrar manifestaciones
puras y radicales del apriorismo o del historicismo y abundan las posturas
intermedias. Lo que normalmente pretenden subrayas los aprioristas no es la
existencia permanente de un conjunto normativo llamado Derecho Civil, sino la
conexión de las históricas formas del mismo con principios que trascienden a un
ordenamiento jurídico determinado. O dicho con mayor precisión, los principios
que normalmente se reconocen como tributarios del Derecho Natural.

A su vez, los historicistas subrayan el aspecto más inmediato de la mutabilidad
y evolución de las instituciones jurídicas, sin que ello implique el
desconocimiento o negación de las relaciones de las mismas con los presupuestos
o principios del denominado Derecho Natural.

\begin{quote}
    \textit{En el medio está la virtud.}. Moral a Nicomano - Aristóteles.
\end{quote}

\begin{quote}
    \textit{Aunque el derecho civil (las instituciones que lo forman) se base
    directamente en el Derecho Natural, el ajuste entre aquel término y su
actual contenido ha sido determinado por circunstancias históricas}. Prof. F. de
Castro.
\end{quote}
\section{La materia propia del derecho civil}
\subsection{Derecho civil como derecho de la persona}

La form codificada del Derecho Civil ha sido el punto crítico de su evolución,
un instrumento. El núcleo central del Derecho Civil viene representado por la
\emph{persona} en sí misma considerada, en su dimensión \emph{familiar} y en sus relaciones
\emph{patrimoniales}, como revela la mera contemplación del índice sistemático
de cualquiera de los Códigos Civiles.

La propia estructura del Código Civil español demuestra lo anterior. Analizando
el contenido del mismo, las materias sobre las que recae su regulación serían
las siguientes:

\begin{itemize}
    \item{Vigencia y efecto de las normas jurídicas}
    \item{Delimitación del ámbito de poder jurídico de las personas y su
        relación con un grupo especial de otras personas que les son
    especialmente próximas por razón del nexo biológico o adoptivo entre ellas
existente (familia)}
    \item{Categorías de bienes que pueden ser objeto de tráfico; poder que las
        personas pueden ostentar sobre dichos bienes; modos de circulación de
    dichos bienes y reglas de transmisión de tales bienes (herencia)}
\end{itemize}

La materia contemplada en el primero de los apartados se refiere a cuestiones
generales de \textit{fuentes del Derecho} y de aplicación y eficacia de las
normas jurídicas que no pueden ser consideradas como exclusivas del Derecho
Civil aunque se integraron en los Códigos Civiles por razones históricas.

Las situaciones típicas que pueden configurarse como contenido de las diversas
formas históricas del Derecho Civil han sido tradicionalmente individualizadas
en la persona, en la familia y en el patrimonio:

\begin{itemize}
    \item{La \textbf{persona} en sí misma considerada, en cuanto sujeto de
        derecho, sin tener en cuenta cualesquiera otros atributos,
    características o situaciones sociales. Cuando se considera a la persona
como el empresario, entrará en juego el derecho mercantil, cuando se considera
como votante, el derecho constitucional, el electoral, etc.}
    \item{La \textbf{familia} como grupo humano básico, necesitado de una
        regulación que encuadre los derechos y deberes recíprocos entre sus
    miembros y de éstos con el resto de la comunidad}
    \item{El \textbf{patrimonio}, conjunto de bienes, derechos y obligaciones de
        cualquier persona con capacidad para adquirir y transmitir bienes.
    También, los instrumentos básicos de intercambio económico (los contratos) y
los mecanismos de transmisión a los familiares a través de la herencia, etc.}
\end{itemize}

Esta partición de contenidos se corresponde con realidades y situaciones de
permanente existencia. El Derecho Civil tiene un marcado carácter social, lo que
conlleva una tensión constante entre el grupo social políticamente organizado,
considerado en su conjunto, y los individuos que lo integran, considerados como
personas en sí mismas.

El problema principal es el marco de libertad y de autonomía del indivíduo
frente al grupo social políticamente organizado. Individuo vs Grupo. También
existe otro problema si la familia se adscribe al individuo o al grupo social.

Se puede conceptuar el Derecho Civil como el \textit{derecho de la personalidad
privada, que se desenvuelve a través de la familia, sirviéndose para sus propios
fines de un patrimonio y asegurando su continuidad a través de la herencia
(Profesor A. Cossio)}.

\section{La codificación y los derechos forales}
\subsection{La codificación en general}

Hacia medidados del siglo XVIII se produce un intento generalizado en toda
Europa de realizar una sistematización del Derecho sobre patrones diversos.
Hasta entonces se había calificado como \emph{codex} o \emph{códice} a un
conjunto de folios en forma de libro, cosido por el lomo y que tenía por objeto
recopilar conjuntos muy heterogéneos de cuestiones o máximas jurídicas sin
criterio determinado. A veces se ordenaban cronológicamente, a veces según la
fuente y otras de forma completamente anárquica.

A partir de este momento la palabra \emph{código} pasa a tener un significado
preciso y representa un ideal a alcanzar para todas las naciones europeas y
muchas sudamericanas.

Dicho periodo puede considerarse abierto con la publicación del Código Civil
francés en 1804y cerrado con la aprobación del Código Civil Alemán de 1896
(Bürgerliches Gesetzbuch o BGB). En ese periodo se racionalizó la materia
jurídica clarificando el sistema jurídico. Se estructura un sistema normativo
único que se aplicará a la generalidad de los ciudadanos (Francia) o súbcitos.

Esta codificación está caracterizada por Claridad, Sistematización, Igualdad
política (teniendo en cuenta las estructuras sociopolíticas del momento) y
Uniformidad Jurídica.

\subsection{La codificación civil española}
\subsubsection{El fracasado proyecto de 1851 y la publicación de las leyes especiales}

El bloque de legislaciones históricas superpuestas era de tal naturaleza que el
fenómeno codificador se consideraba como un necesario punto de partida para la
construcción de la España del siglo XIX. Todas las constituciones decimonónicas
incliyen dentro de su articulado la aspiración a la codificación del Derecho
patrio.

Dicha aspiración resultó relativamente pacífica en relación con algunas materias
como el Derecho Mercantil, que ya tomara cuerpo codificado en 1829. La
codificación civil, sin embargo, resultó mucho más problemática por la tensión
existente entre el Derecho Común y los Derechos Forales, entre otras razones.

El verdadero punto de partida de la codificación civil española viene
representado por el \textbf{Proyecto de Código Civil de 1851}, comunmente
llamado \emph{Proyecto isabelino}. Dicho trabajo nace como uno de los primeros
frutos de la recién creada Comisión General de Códigos, según Real Decreto del
19 de agosto de 1842, que actualmente se llama Comisión General de Codificación
y que dependía del Ministerio de Justicia.

Dicho proyecto se caracterizaba por dos características que provocaron su
fracaso:

\begin{itemize}
\item{Ser notoriamente afrancesado, siguiendo muy de cerca los patrones propios del Código Civil francés}
\item{Unificaba la legislación civil española, eliminando los Derechos Forales}
\end{itemize}

La necesidad de encarar la actualización de la legislación civil era evidente y
en las decadas siguientes se fue desgranando paulatinamente la aprobación de
leyes importantísimas que deberían haber sido incorporadas al Código Civil como
la Ley Hipotecaria, la Ley de Matrimonio Civil, la Ley de Registro Civil o la
Ley de Propiedad Intelectual. Tales leyes reciben la adjetivación de
\textbf{especiales} en cuanto se considera que los aspectos comunes o
fundamentales de ellas deberían ser recogidos en el Código Civil.

La calificación de \emph{especiales} resalta \emph{ab initio (desde el principio)} la idea de que son
leyes extracodificadas, leyes especiales en contraposición a las leyes generales
que están en el Código Civil.

\subsubsection{El código civil}

En enero de 1880, el Ministro de Justicia Álvarez Bugallal, insta a la comisión
de Códigos a que en el plazo de un año redacte el Código sobre la base del
Proyecto de 1851, incorporando al efecto juristas de los territorios forales.

Al mes siguiente, Manuel Alonso Martínez trata de impulsar la tarea codificadora
recurriendo a la idea de la \textbf{Ley de Bases (los principios, fundamentos o
presupuestos fundamentales)}. Se presentaría a las Cámaras
legislativas una Ley en la que se contuvieran los principios y fundamentos a
desarrollar en el Código Civil, mientras que en la redacción del texto
articulado quedaría encomendada a los organismos técnicos. Rechazado en el
Congreso, se vio obligado a presentarlo ante la Cámara por libros, esto es, por
partes.

Con ligeros retoques, volvió a insistir en la idea de la Ley de Bases en 1885,
la cual tras la consiguiente tramitación parlamentaria fue aprobada como Ley en
1888.

Siguiendo las bases establecidas por dicha Ley, la Comisión de Códigos llevo a
cabo su misión de redactar el texto articulado de forma algo atropellada y
con recortes de última hora.  El texto del Código Civil se publicó finalmente en la \emph{Gaceta}
(hoy B.O.E.) de 25, 16 y 27 de julio de 1889.

\subsubsection{La evolución posterior de la legislación civil}

Desde su publicación hasta la fecha, el texto articulado del Código Civil ha
sido objeto de numerosas reformas, aunqeu la mayor parte del mismo ha resistido
bien el paso del tiempo y sigue fiel a los textos originarios.

Entre las reformas habidas, las fundamentales y más profundas son las debidas a
la necesidad de adecuar el contenido del Código a la Constitución española de
1978, sobre todo en lo referente a la igualdad entre hombres y mujeres y a la
igualdad entre hijos matrimoniales y extramatrimoniales.

Otro bloque de disposiciones modificativas del texto articulado del Código Civil
responde a razones meramente técnicas. Sin duda la más importante es la Ley de
1973, en cuya virtud se dio nueva redacción al \emph{Título preliminar} del
Código Civil.

\subsection{Los derechos forales y la llamada cuestión foral}

Durante los siglos XVIII y XIX existía en España una cierta diversidad de
regulaciones civiles pues Aragón, Navarra, Mallorca, Cataluña y las Provincias
Vascongadas mantenían reglas propias en material civil, sobre todo en lo
referido a la familia y a la herencia. Sin embargo, la codificación requería la
unificación legislativa en toda España, a lo que los juristas forales se
negaban. Este fue uno de los motivos por los que no salió adelante el
\emph{proyecto isabelino}, al pretender abrogar (derogar) los Derechos Forales.

Cuando se aprueba el Código Civil a finales del siglo XIX no hay consenso y nade
la "cuestión foral". Una vez aprobado el Código Civil, este se aplica a la mayor
parte del territorio nacional, mientras que en los Territorios Forales (Aragón,
Navarra, Mallorca, Cataluña, Provincias Vascongadas y se añade Galicia en 1880)
rigen disposiciones de naturaleza civil propias, de diferente alcance, extensión
y significado.

\subsection{Las distintas soluciones de la cuestión foral}
\subsubsection{Ley de bases y redacción originaria del Código civil, la técnica prevista de los Apéndices}

La división entre el derecho civil común y los derechos forales ha estado viva
durante todo el periodo y sigue latente en nuestros días.

El artículo 5 de la Ley de Bases del Código Civil era suficientemente explícito:

\begin{quote}
    \textit{las provincias y territorios en que subsiste Derecho foral, lo
    conservarán por ahora en toda su integridad, sin que sufra alteración su
actual régimen jurídico por la publicación del Código, que regirá tan solo como
supletorio en defecto del que lo sea aen cada una de aquellas por sus leyes
especiales...}
\end{quote}

El artículo 6 menciona que:

\begin{quote}
    \textit{el Gobierno, oyendo a la Comisión de Códigos, presentará a las
        Cortes, en uno o varios Proyectos de Ley, los apéndices del Código Civil
    en los que se contengan las instituciones forales}
\end{quote}

De todo esto se pueden sacar las siguientes concluiones:

\begin{itemize}
\item{El Código Civil respeta y garantiza los Derechos Forales. Es valedor del
    régimen normativo foral}
\item{Los Códigos Forales se consideran provisionales}
\item{Los Derechos Forales no son sistemas paralelos al Código Civil, sino
    complementarios; tienen \emph{Carácter Apendicular}. La intención era
conseguir un Código Civil General (único). Sólo se aprobó en 1926 el apéndice de
Aragón}
\end{itemize}

La vigencia normativa de la Ley de Bases del Código Civil se agotó una vez
publicado este, no obstante tiene un gran valor como instrumento para
interpretar el Código Civil.

\subsubsection{Las diversas compilaciones forales}

En 1946 en Zaragoza se celebró un un Congreso Nacional de Derecho Civil para
tratar el asunto de integrar el derecho común y los derechos civiles forales.

En dicho congreso se obtuvo un relativo consenso respecto de los siguientes
puntos fundamentales:

\begin{itemize}
\item{Llevar a cabo una recopilación de las instituciones forales o
    territoriales, no sólo las vigentes sino también las no decaídas por el uso}
\item{Tratar de determinar el substrato común para elaborar un Código Civil
    general que se desarrollaría en un nuevo Congreso Nacional (que nunca se
celebró)}
\end{itemize}

Durante el gobierno de Franco, entre 1959 y 1973 se aprobaron en las Cortes Generales las distintas
Compilaciones forales.

\paragraph{Compilación de Derecho Civil Foral de Vizcaya y Álava}

Ley de 30 de julio de 1959. No es aplicable en todo el territorio, sólo en el
campo. Potencia el principio supremo de la concentración patrimonial en torno al
caserío familiar.

\paragraph{Compilación del Derecho Civil Especial de Cataluña}

Ley de 21 de julio de 1960. Rige en toda Caraluña aunque alguna disposición es
de carácter local. Tiene 334 artículos en 4 libros: Familia (separación de
bienes matrimonial), Sucesiones (herencias y fideicomisos), Derechos reales
(censos), Obligaciones y contratos y de la prescripción.

\paragraph{Compilación del Derecho Civil Especial de Baleares}

Ley de 19 de abril de 1961. Matriz común con la catalana pero con menor casuismo
y mayor respeto del Derecho propio. Sólo es aplicable en las islas de Mallorca,
Menorca y, en menor medida, en Ibiza y Formentera. Se divide en tres libros.

\paragraph{Compilación del Derecho Civil Especial en Galicia}

Ley de 2 de diciembre de 1963. Galicia no tuvo derecho propio hasta 1880. Dicha
compilación es breve (93 artículos), centrada en 5 títulos del estatuto
gregario:

\begin{enumerate}
\item{Foro, subforo y otros gravámenes análogos}
\item{Compañía familiar gallega}
\item{Aparcería}
\item{Derecho de labrar y poseer}
\item{Formas especiales de comunidad}
\end{enumerate}

\paragraph{Compilación del Derecho Civil de Aragón}

Ley de 8 de abril de 1967. Una de las más técnicas y mejor redactadas. Es un
régimen distinto al castellano con peculiaridades familiares y sucesorias:
Comunidad conyugal de bienes y ganancias, viudedad foral aragonesa (protege al
viudo).

\paragraph{Compilación del Derecho Civil Foral de Navarra}

Ley de prerrogativa de 1 de marzo de 1973, también llamada \emph{Fuero Nuevo de
Navarra}. La compilación no se encuentra dividida por artículos sino por Leyes.
Es extensa y variada, difícil de resumir y repite artículos del mismo Código
Civil.

\subsection{Relaciones entre el derecho civil general y los derechos civiles forales tras la Constitución}
\subsubsection{El artículo 149.1.8 de la Constitución}

Las compilaciones debían ser un paso previo a la unificación del Código Civil,
pero con la Constitución de 1978 todo cambió. Se consolidan las compilaciones y
faculta a las Comunidades Autónomas en las que existan derechos forales o
especiales para la conservación, modificación y desarrollo de los mismos.

Dichas premisas se encuentran establecidas en el artículo 149.8.1 de la
Constitución Española, el cual ha originado un fortísimo debate entre los
civilistas actuales sobre su alcance y significado:

\begin{quote}
    El Estado tiene competencia exclusiva sobre la legislación civil, sin
    perjuicio de la conservación, modificación y desarrollo por las Comunidades
    Autónomas de los derechos civiles forales o especiales, allí donde existan.
    En todo caso, las reglas relativas a la aplicación y eficacia de las normas
    jurídicas, relaciones jurídico-civiles relativas a las formas de matrimonio,
    ordenación de los registros e instrumentos públicos, bases de las
    oblicaciones contractuales, normas para resolver los conflictos de leyes y
    determinación de las fuentes del Derecho, con respeto en este último caso a
    las normas de Derecho foral o especial.
\end{quote}

La mera lectura de dicho precepto evidencia su complejidad. La tensión, pues,
entre Derecho civil común y Derechos civiles forales o especiales sigue
irresuelta también tras la constitución.

No obstante hay cosas que sí quedan claras:

\begin{itemize}
    \item{El precepto es muy complejo}
    \item{Siguen siendo calificadas de forales o especiales}
    \item{Sólo aquellas CCAA con compilaciones previas pueden conservarlas:
        perspectiva historicista}
\end{itemize}

Posturas o interpretaciones básicas sobre el tema:

\begin{enumerate}
\item{El concepto de Derecho foral debe ser abandonado tras la CE. Las CCAA
    pueden legislar sobre cualquier materia de Derecho civil. Foralistas
(catalanes)}
\item{Otros civilistas entienden que los límites a la conservación, modificación
    y desarrollo de los derechos civiles forales o especiales, vendrían
representados por el contenido normativo de las respectivas Compilaciones enel
momento de aprobarse la Constitución Española}
\item{Algún foralista propugna que el carácter particular de los Derechos
    forales sólo puede identificarse a través de los principios inspiradores que
le son propios.}
\item{Para otros, el límite constitucional de desarrollo de Derecho civil foral
    ha de identificarse con las instituciones características y propias de los
territorios forales que, tradicionalmente, han sido reguladas de forma distinta
por el Derecho común y por los Derechos forales (a juicio del autor de esta
obra, la posición técnicamente más correcta es la última)}
\end{enumerate}

Aún queda pendiente que el Tribunal Constitucional siente doctrina.

\subsubsection{La actualización de las compilaciones forales y de los derechos forales}

Uno de los elementos políticos de "diferenciación regional" radica en el
desarrollo de los Derechos forales a los que se refiere el Artículo 149 de la
Constitución Española. Una vez aprobados los Estatutos de Autonomía, los órganos
legislativos de las Comunidades Autónomas que tenían Derecho foral o especial,
han comenzado a desarrollar el Derecho privado propio de los antiguos
territoriales forales. Esto se ha plasmado sobre todo a mediados de los 80 en
leyes autonómicas cuyo objetivo básico ha sido doble:

\begin{itemize}
\item Constitucionalizar el contenido de las compilaciones adecuándolo a los nuevos
principios de igualdad entre hombres y mujeres y entre hijos matrimoniales y no
matrimoniales que marca la Constitución Española
\item Evidenciar que las compilaciones dejan de ser leyes nacionales, ya que con la 
    Constitución Española las materias reguladas por ellas corresponden a las 
    Comunidades Autónomas.
\end{itemize}

Pasada esta primera etapa, la mayoría de las Comunidades Autónomas se ha
preocupado por la actualización de las Compilaciones, siempre como \emph{Derecho
foral o especial}. Sin embargo, la diferenciación normativa va en aunmento.

Entre todas las Comunidades Autónomas con competencia foral sobresale sin duda
la actividad legislativa catalana.

\chapter{Tema 2}
% Capítulo 3: epígrafes 1, 2, 5, 6 y 7
\section{Estructura general y clases de las normas jurídicas}
El derecho es, fundamentalmente, un instrumento que ordena la convivencia
social. El Derecho es un conjunto de reglas que trata de dar solución a los
conflictos sociales. Esas reglas se llaman \textbf{normas jurídicas} y están
caracterizadas por:

\begin{itemize}
\item{Obligatoriedad: tienen que ser tenidas en cuenta por los ciudadanos}
\item{Coercibilidad: pueden imponerse por la fuerza}
\end{itemize}

\subsubsection{Norma jurídica y disposición normativa}

La \textbf{norma jurídica} (lo que se dice) es un mandato jurídico con eficacia
social organizadora (de Castro). Según García Amigo, es un precepto regulador de la conducta de los
ciudadanos, obligatorio y coercible, inspirado en un criterio de justicia. No
todas las normas jurídicas tienen detrás una disposición normativa o texto
concreto, a veces se basan en costumbres vinculantes.

La \textbf{disposición normativa} (el contenedor publicado) sirve de vehículo a
las normas jurídicas. Están escritas, pero no todo lo escrito es vehículo de
Normas Jurídicas, a veces es necesario combinar varias Disposiciones Normativas
para llegar a una norma jurídica.

\subsubsection{Disposiciones completas e incompletas}

Las \textbf{disposiciones jurídicas completas}, también llamadas autónomas, son
las portadoras de una norma jurídica completa.

Las \textbf{disposiciones jurídicas incompletas}, o auxiliares-fragmentarias,
pueden ser combinadas con otras del mismo carácter y deducir un mandato
normativo o bien ser combinadas con otras disposiciones completas para precisar
detalles o limitar su ámbito.

Las disposiciones jurídicas incompletas pueden ser de los siguientes tipos:

\begin{itemize}
\item{Las que aclaran conceptos o ideas fijadas en otras normas}
\item{Las de remisión o reenvío de una normativa que ya existe: por ejemplo,
    aplicar las normas de compraventa a la permuta}
\item{Las que concretan la aplicación y eficacia de una verdadera norma (son
    parte de la norma))}
\end{itemize}

\subsubsection{La estructura de la norma: supuesto de hecho y consecuencia
jurídica}

La norma jurídica se estructura en dos elementos fundamentales, de carácter
general y abstracto:

En primer lugar, el \textbf{supuesto de hecho} o \emph{supuesto normativo}, es
decir, la realidad social a regular, las situaciones fácticas a las que están
dirigidas las normas (actos humanos, hechos naturales, etc).

Por otro lado, la \textbf{consecuencia jurídica}, el mandato o precepto de
carácter prohibitivo o permisivo. Está en el campo del \emph{deber ser},
contienen una valoración del conflicto de intereses y atribuye derechos y/o
obligaciones.

\subsubsection{Abstracción y generalidad de la norma}

Las normas pueden tener un carácter general o abstracto. Decimos que una norma
es de \textbf{carácter general} cuando no está dirigida a alguien en concreto, sino a un
grupo de personas (por ejemplo, a empleados) o a la colectividad (por ejemplo el
Derecho Penal). Por contra, una norma es de \textbf{carácter abstracto} cuando
contempla un supuesto tipo, genérico y que luego se adecúa a cada caso concreto
mediante matizaciones de los abogados y jueces.

\subsubsection{Normas de derecho común y derecho especial}

Las normas de \textbf{derecho especial} regulan cosas concretas o a un sector
determinado o colectivo. Ejemplos de estas normas son el Derecho Mercantil o el
Derecho Laboral.

Las normas de \textbf{derecho común}, es decir, el Derecho Civil, cumple una
función supletoria respecto al Derecho Especial. El artículo 4.3 del Código
Civil establece que \begin{quote}las disposiciones del Código Civil se aplicarán
como supletorias en las materias regidas por otras leyes\end{quote}.

\subsubsection{Normas de derecho general y derecho particular}

Las normas \textbf{de derecho general} se aplican en todo el territorio
nacional.

Las normas \textbf{de derecho particular} o \textit{especiales} son aquellas aplicables a territorios
más reducidos.

\begin{itemize}
\item{Comarcas: costumbres}
\item{Regiones: derechos forales}
\item{Comunidades autónomas}
\end{itemize}

\subsubsection{La imperatividad del derecho: normas imperativas y dispositivas}

Las \textbf{normas imperativas} son aquellas \emph{ius cogens} (no se pueden alterar). El
mandato normativo no permite modificación por los particulares, no se pueden
derogar o sustituir, como por ejemplo la Ley Fiscal. Es Derecho inderogable.

Las \textbf{normas dispositivas} son los mandatos normativos que pueden ser
sustituidos por los interesados si así lo quieren. La norma jurídica desempeña
una función supletoria cuando los particulares no han establecido otras normas.
Existe autonomía privada, pero si no se ejerce se aplica el Derecho Supletorio.

\section{Génesis de las normas jurídicas: fuentes del ordenamiento jurídico español}
\subsection{El planteamiento civilista y el significado de la expresión "fuentes del derecho"}
El artículo 1.1 del Código Civil indica que \begin{quote} las fuentes del
    ordenamiento jurídico español son la Ley, la Costumbre y los Principios
Generales del Derecho.\end{quote}. Tanto la ley como la costumbre y los
principios generales del Derecho son los vehículos portadores de las normas
jurídicas desde el punto de vista formal.

Al hablar de fuentes del Derecho se está haciendo referencia al \emph{cómo} se
generan las normas jurídicas, al modo de producción de las normas jurídicas en
un doble sentido:

\begin{description}
\item[Formal] en cuanto modos o formas de manifestación del Derecho, bien a
    través de ley o a través de costumbre: \emph{fuente en sentido formal}
\item{Material} al considerar las instituciones o grupos sociales que tienen
    reconocida capacidad normativa (las Cortes Generales, el Gobierno, etc).
    \emph{fuente en sentido material}
\end{description}

Otros posibles significados de fuentes del derecho:

\begin{itemize}
\item{Legitimadora: al preguntarse sobre el por qué último del derecho, se suele
    hablar de \textit{fuente legitimadora} del mismo, en el sentido de que el
Ordenamiento Jurídico se asienta en las ideas comunes sobre la Justicia como
último principio inspirador}.
\item{Conocimientos: también se habla de \textit{fuentes del conocimiento del Derecho} para hacer referencia al Instrumental del que se sirven los juristas para identificar el Derecho positivo de una determinada colectivdad y un preciso momento histórico.}
\end{itemize}

\subsection{La consideración del tema desde el prisma constitucional}

Se ha afirmado que el título preliminar del Código Civil tiene \textem{valor constitucional}, pero no cabe duda de que mayor valor constitucional tiene el propio artículo 1.2 de la Constitución: \textem{la soberanía nacional reside en el pueblo español, del que emanan los poderes del Estado}.

Al ser la constitución una derivación de dicha soberanía nacional (ya que el pueblo español la aprobó por referéndum), es precisamente ella la que delimita el verdadero sistema normativo y expresa las capacidades legislativas de las instituciones.

Conforma a ella, tanto el Estado central (Cortes Generales), como las Comunidades Autónomas (Parlamentos autonómicos o Asambleas legislativas) pueden dictar leyes en sus respectivos territorios.

La Constitución dictamina que la costumbre sólo tiene valor en el Código Civil, no en otras ramas del derecho.

La Constitución tiene además una visión más internacional del sistema normativo y las fuentes del derecho. El artículo 93 de la Constitución Española prevé que \textem{se podrán autorizar la celebración de tratados por los que se atribuya a una organización o institución internacional el ejercicio de competencias derivadas de la Constitución Española}. Este artículo se redactó pensando en el ingreso de españa en la CEE.