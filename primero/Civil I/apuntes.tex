\documentclass[a4paper,12pt]{report}
\usepackage[utf8]{inputenc}
\usepackage[activeacute,spanish]{babel}
% Remove hyphenation
\tolerance=1
\emergencystretch=\maxdimen
\hyphenpenalty=10000
\hbadness=10000
\begin{document}
\title{Derecho Civil I}
\author{Alberto Corsín Lafuente}
\date{\today}
\maketitle
\tableofcontents
\chapter{Tema I}
\section{El concepto de derecho civil}
\subsection{Introducción}

No se puede dar una definición precisa del derecho que satisfaga las diversas
líneas del pensamiento y resulte admisible para la generalidad de autores. Ya
desde Immanuel Kant (s XIX) los juristas buscan una definición del concepto de
Derecho.

Para los tratadistas franceses del siglo XIX el Derecho Civil venía representado
por el Código Civil de 1804 (escuela exegesis). Los autores alemanes de
comienzos del siglo XX lo identificaron con el contenido del Burgerliches
Gesetzbuch, código alemán de 1896 (BGB) vigente desde principios de 1990. Sin
embargo, ninguno de los dos grupos llegó a formular una definición de aquello
que había de ser objeto de sus normas.

Semejante identificación material entre los Códigos y el Derecho Civil resulta
inaceptable de forma mayoritaria en los inicios del siglo XXI por razones
evidentes. Por importantes que puedan ser los Códigos Civiles, es obvio que el
Derecho Civil no quedó petrificado en ellos, sino que la legislación posterior y
las coordenadas sociales en general han diversificado su contenido, cuando no
han roto algunos de los principios básicos en que se asentaba la regulación
codificada. El Derecho Civil es algo más que un código.

\subsection{El planteamiento historicista y concepción apriorística}

Las teorías contemporáneas se dividen en historicistas y apriorísticas.

Las \textbf{teorías historicistas} recalcan el carácter histórico y evolutivo
del Derecho Civil. Recalcan la mutabilidad y evolución de las instituciones
jurídicas sin negar su relación con el Derecho Natural. Según estas teorías el
Derecho Civil es distinto en cada época.

Las \textbf{teorías apriorísiticas} o racionalistas puros hacen especial
hincapié en la nota de permanencia del derecho civil desde una perspectiva meta
histórica. Según esta línea de pensamiento los principios generales o
instituciones del Derecho Civil son siempre iguales, constantes. Cambia la forma
de regular, no los principios y se recalca la conexión entre Derecho Civil y
Derecho Natural. Por ejemplo: el principio de libertad de la persona, principio
de autonomía de la voluntad, etc. Estas teorías son minoritarias.

\subsection{La superación de la contraposición entre historicismo y apriorismo}

Doctrinalmente hablando resulta sumamente difícil encontrar manifestaciones
puras y radicales del apriorismo o del historicismo y abundan las posturas
intermedias. Lo que normalmente pretenden subrayas los aprioristas no es la
existencia permanente de un conjunto normativo llamado Derecho Civil, sino la
conexión de las históricas formas del mismo con principios que trascienden a un
ordenamiento jurídico determinado. O dicho con mayor precisión, los principios
que normalmente se reconocen como tributarios del Derecho Natural.

A su vez, los historicistas subrayan el aspecto más inmediato de la mutabilidad
y evolución de las instituciones jurídicas, sin que ello implique el
desconocimiento o negación de las relaciones de las mismas con los presupuestos
o principios del denominado Derecho Natural.

\begin{quote}
    \textit{En el medio está la virtud.}. Moral a Nicomano - Aristóteles.
\end{quote}

\begin{quote}
    \textit{Aunque el derecho civil (las instituciones que lo forman) se base
    directamente en el Derecho Natural, el ajuste entre aquel término y su
actual contenido ha sido determinado por circunstancias históricas}. Prof. F. de
Castro.
\end{quote}
\section{La materia propia del derecho civil}