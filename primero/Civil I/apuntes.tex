\documentclass[a4paper,12pt]{report}
\usepackage[utf8]{inputenc}
\usepackage[activeacute,spanish]{babel}
% Remove hyphenation
\tolerance=1
\emergencystretch=\maxdimen
\hyphenpenalty=10000
\hbadness=10000
\begin{document}
\title{Derecho Civil I}
\author{Alberto Corsín Lafuente}
\date{\today}
\maketitle
\tableofcontents
\chapter{Tema I}
\section{El concepto de derecho civil}
\subsection{Introducción}

No se puede dar una definición precisa del derecho que satisfaga las diversas
líneas del pensamiento y resulte admisible para la generalidad de autores. Ya
desde Immanuel Kant (s XIX) los juristas buscan una definición del concepto de
Derecho.

Para los tratadistas franceses del siglo XIX el Derecho Civil venía representado
por el Código Civil de 1804 (escuela exegesis). Los autores alemanes de
comienzos del siglo XX lo identificaron con el contenido del Burgerliches
Gesetzbuch, código alemán de 1896 (BGB) vigente desde principios de 1990. Sin
embargo, ninguno de los dos grupos llegó a formular una definición de aquello
que había de ser objeto de sus normas.

Semejante identificación material entre los Códigos y el Derecho Civil resulta
inaceptable de forma mayoritaria en los inicios del siglo XXI por razones
evidentes. Por importantes que puedan ser los Códigos Civiles, es obvio que el
Derecho Civil no quedó petrificado en ellos, sino que la legislación posterior y
las coordenadas sociales en general han diversificado su contenido, cuando no
han roto algunos de los principios básicos en que se asentaba la regulación
codificada. El Derecho Civil es algo más que un código.

\subsection{El planteamiento historicista y concepción apriorística}

Las teorías contemporáneas se dividen en historicistas y apriorísticas.

Las \textbf{teorías historicistas} recalcan el carácter histórico y evolutivo
del Derecho Civil. Recalcan la mutabilidad y evolución de las instituciones
jurídicas sin negar su relación con el Derecho Natural. Según estas teorías el
Derecho Civil es distinto en cada época.

Las \textbf{teorías apriorísiticas} o racionalistas puros hacen especial
hincapié en la nota de permanencia del derecho civil desde una perspectiva meta
histórica. Según esta línea de pensamiento los principios generales o
instituciones del Derecho Civil son siempre iguales, constantes. Cambia la forma
de regular, no los principios y se recalca la conexión entre Derecho Civil y
Derecho Natural. Por ejemplo: el principio de libertad de la persona, principio
de autonomía de la voluntad, etc. Estas teorías son minoritarias.

\subsection{La superación de la contraposición entre historicismo y apriorismo}

Doctrinalmente hablando resulta sumamente difícil encontrar manifestaciones
puras y radicales del apriorismo o del historicismo y abundan las posturas
intermedias. Lo que normalmente pretenden subrayas los aprioristas no es la
existencia permanente de un conjunto normativo llamado Derecho Civil, sino la
conexión de las históricas formas del mismo con principios que trascienden a un
ordenamiento jurídico determinado. O dicho con mayor precisión, los principios
que normalmente se reconocen como tributarios del Derecho Natural.

A su vez, los historicistas subrayan el aspecto más inmediato de la mutabilidad
y evolución de las instituciones jurídicas, sin que ello implique el
desconocimiento o negación de las relaciones de las mismas con los presupuestos
o principios del denominado Derecho Natural.

\begin{quote}
    \textit{En el medio está la virtud.}. Moral a Nicomano - Aristóteles.
\end{quote}

\begin{quote}
    \textit{Aunque el derecho civil (las instituciones que lo forman) se base
    directamente en el Derecho Natural, el ajuste entre aquel término y su
actual contenido ha sido determinado por circunstancias históricas}. Prof. F. de
Castro.
\end{quote}
\section{La materia propia del derecho civil}
\subsection{Derecho civil como derecho de la persona}

La form codificada del Derecho Civil ha sido el punto crítico de su evolución,
un instrumento. El núcleo central del Derecho Civil viene representado por la
\emph{persona} en sí misma considerada, en su dimensión \emph{familiar} y en sus relaciones
\emph{patrimoniales}, como revela la mera contemplación del índice sistemático
de cualquiera de los Códigos Civiles.

La propia estructura del Código Civil español demuestra lo anterior. Analizando
el contenido del mismo, las materias sobre las que recae su regulación serían
las siguientes:

\begin{itemize}
    \item{Vigencia y efecto de las normas jurídicas}
    \item{Delimitación del ámbito de poder jurídico de las personas y su
        relación con un grupo especial de otras personas que les son
    especialmente próximas por razón del nexo biológico o adoptivo entre ellas
existente (familia)}
    \item{Categorías de bienes que pueden ser objeto de tráfico; poder que las
        personas pueden ostentar sobre dichos bienes; modos de circulación de
    dichos bienes y reglas de transmisión de tales bienes (herencia)}
\end{itemize}

La materia contemplada en el primero de los apartados se refiere a cuestiones
generales de \textit{fuentes del Derecho} y de aplicación y eficacia de las
normas jurídicas que no pueden ser consideradas como exclusivas del Derecho
Civil aunque se integraron en los Códigos Civiles por razones históricas.

Las situaciones típicas que pueden configurarse como contenido de las diversas
formas históricas del Derecho Civil han sido tradicionalmente individualizadas
en la persona, en la familia y en el patrimonio:

\begin{itemize}
    \item{La \textbf{persona} en sí misma considerada, en cuanto sujeto de
        derecho, sin tener en cuenta cualesquiera otros atributos,
    características o situaciones sociales. Cuando se considera a la persona
como el empresario, entrará en juego el derecho mercantil, cuando se considera
como votante, el derecho constitucional, el electoral, etc.}
    \item{La \textbf{familia} como grupo humano básico, necesitado de una
        regulación que encuadre los derechos y deberes recíprocos entre sus
    miembros y de éstos con el resto de la comunidad}
    \item{El \textbf{patrimonio}, conjunto de bienes, derechos y obligaciones de
        cualquier persona con capacidad para adquirir y transmitir bienes.
    También, los instrumentos básicos de intercambio económico (los contratos) y
los mecanismos de transmisión a los familiares a través de la herencia, etc.}
\end{itemize}

Esta partición de contenidos se corresponde con realidades y situaciones de
permanente existencia. El Derecho Civil tiene un marcado carácter social, lo que
conlleva una tensión constante entre el grupo social políticamente organizado,
considerado en su conjunto, y los individuos que lo integran, considerados como
personas en sí mismas.

El problema principal es el marco de libertad y de autonomía del indivíduo
frente al grupo social políticamente organizado. Individuo vs Grupo. También
existe otro problema si la familia se adscribe al individuo o al grupo social.

Se puede conceptuar el Derecho Civil como el \textit{derecho de la personalidad
privada, que se desenvuelve a través de la familia, sirviéndose para sus propios
fines de un patrimonio y asegurando su continuidad a través de la herencia
(Profesor A. Cossio)}.

\section{La codificación y los derechos forales}
\subsection{La codificación en general}

Hacia medidados del siglo XVIII se produce un intento generalizado en toda
Europa de realizar una sistematización del Derecho sobre patrones diversos.
Hasta entonces se había calificado como \emph{codex} o \emph{códice} a un
conjunto de folios en forma de libro, cosido por el lomo y que tenía por objeto
recopilar conjuntos muy heterogéneos de cuestiones o máximas jurídicas sin
criterio determinado. A veces se ordenaban cronológicamente, a veces según la
fuente y otras de forma completamente anárquica.

A partir de este momento la palabra \emph{código} pasa a tener un significado
preciso y representa un ideal a alcanzar para todas las naciones europeas y
muchas sudamericanas.

Dicho periodo puede considerarse abierto con la publicación del Código Civil
francés en 1804y cerrado con la aprobación del Código Civil Alemán de 1896
(Bürgerliches Gesetzbuch o BGB). En ese periodo se racionalizó la materia
jurídica clarificando el sistema jurídico. Se estructura un sistema normativo
único que se aplicará a la generalidad de los ciudadanos (Francia) o súbcitos.

Esta codificación está caracterizada por Claridad, Sistematización, Igualdad
política (teniendo en cuenta las estructuras sociopolíticas del momento) y
Uniformidad Jurídica.